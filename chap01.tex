\chapter{Relationship extraction intro} \todo{změnit název}

\section{Terminology}
Terminology in NLP subtasks is often not exact or non-standardized. We will attempt to introduce the most important concepts for our work as exactly as possible while respecting the terms that seem to be already established. \todo{hloupá věta}

\todo{někam zapracovat příklad na větě, jak je nejdřív najití rozštíření a tak všechno}

\defineterm{Relation} in this context is an abstraction of a semantic relation, for example a father relation. Relation is of a type (father) is binary (between the son and the father) and oriented (the father and the son are not interchangeable), and describes the relationship between a subject (the son) and an object (the father). We will use the term relation as an equivalent for its type and the term relationship for an instance of the relation. 

\todo{obrázek pro příklad v dalším odstavci}

\defineterm{Subject} and \defineterm{Object}. The subject is the first argument of a relation, the object is the second. In the sentence \vuvozovkach{Albus Severus is Harry Potters's son.} a relation of type\relationtype{son} is captured, the subject is Harry and the object is Albus Severus. The reasoning for this choice of direction is as follows: suppose we are gathering information about Harry, then we would probably have both the information that his son is Albus Severus and that his father is James. So we are gathering information about the subject (Harry Potter), even though in most sentences Harry is likely to be the grammatical object: \vuvozovkach{James is Harry's father.} We will use the notation \relation{relation}{subject}{object}: \relation{son}{Harry Potter}{Albus Severus Potter}. \todo{will we, lepší vzhled}

Both the subject and the object can generally be any word or sequence of words that represent concepts that have the ability to form relations. In some cases subjects, objects, or both are limited to entities or named entities.

\defineterm{Named entity} \todo{ukradeno z wiki, ocitovat nebo ukráct od jinud}  is a real-world object, such as persons, locations, organizations, products, etc., that can be denoted with a proper name. It can be abstract or have a physical existence. Named entities can simply be viewed as entity instances (e.g., New York City is an instance of a city). Sometimes, numeric data is considered in this category as well (for example by Named Entity Recognition tools). An \defineterm{entity} is a named entity whose proper name is unknown or unimportant but still is an entity instance. \todo{definice kruhem}

\defineterm{Relation inventory} is the set of relations, that are considered valid for a given dataset or model.

\defineterm{Positive relation mention} is a sentence, that captures a relationship: a relation together with a tagged subject and object. We will omit the word positive unless we want to emphasize the fact.

\defineterm{Negative mention} is close to relation mention in the sense that it is a sentence with tagged subject and object, but the relation type is one of these types: \todo{odrážky}
\begin{itemize}
\item \relationtype{other} - human annotator would classify a relation, that is not in inventory.
\item \relationtype{no relation} - in this case, human annotators should feel an absence of a relationship between a subject and an object. \todo{doplnit příklady, včetně vyloženě ne příkladu}



\end{itemize}

\relationtype{No relation} comes with difficulties. Since there is no semantic relationship between subject and object, it makes it harder to choose subject-object pairs. It is probably desirable to have subject-object pairs, that could be related in a different sentence.

\defineterm{Relationship Extraction} \todo{znímit entity z názvu}

\defineterm{Lemma} \todo{A form from a lexeme chosen by convention (e.g.,
nominative singular for nouns, infinitive for verbs) to represent that
set.
Also called the canonical/base/dictionary/citation form. For every
form, there is a corresponding lemma}

\defineterm{Lexeme}\todo{An abstract entity; the set of all forms related by
inflection (but not derivation).
}

\defineterm{Noun phrase}
\section{Czech language}
One of the objectives of this thesis is to work with Czech language, therefore we find it useful to make some notes on Czech (for non-Czech speaking readers). Czech is a Slavic language with rich morphology and relatively free word order. Most of Czech morphology can be treated with a morphological analyzer, still, it might be useful to have a better understanding of the language we will work with.

\subsection{Inflection}
In Czech, nouns, adjectives, pronounce and numerals are declined. The inflection expresses (not necessarily unambiguously) one of seven cases and a number (singular or plural). Any inflected word in Czech has a grammatical gender, for words, that have natural gender, those two genders align: \vuvozovkach{žena} (\textit{woman}) is feminine and \vuvozovkach{muž} (\textit{man}) is masculine. The inflection of each declinable word follows a pattern. This all means that a single word (lemma) can have a lexeme of size \todo{kolik?}

Verbs are conjugated, the conjugation expresses person, numeral, tense, voice, and mode. Verbs follow one of 14 patterns. An average Czech either finds the theory about Czech verbs and tenses confusing or is even unaware of the existence of the verb patterns, some verbs tend to be used in a grammatically incorrect forms even in the official language.

An important aspect of declension for us is agreement. In English, subject and verb agree (limited just to the third person). In Czech, subject and verb also agree, but in noun phrases there needs to be an agreement as well. \todo{příklad: toho, jak je nějaká noun phrase, počet lexémů, počet validních .. } \todo{odkaz dopředu, kde řeším, jak matchovat}

\subsection{Free word order}

\todo{https://www.aclweb.org/anthology/P14-5003.pdf, statistiky o češtině}