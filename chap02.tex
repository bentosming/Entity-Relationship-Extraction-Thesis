\chapter{Existing Datasets}
\label{chap:datasets}

In this chapter, we overview three well-known datasets related to Entity Relationship Extraction. We start with supervised datasets (SEMEVAL 2010 task 8 and TACRED), then we focus on distant supervision.


\section{SEMEVAL 2010 Task 8 Dataset}
The SemEval-2010 Task 8 dataset (S10T8) was introduced in SemEval-2010 Task 8: Multi-Way Classification of Semantic Relations Between Pairs of Nominals \citep{semeval}. We summarize how S10T8 was created and present additional information from the article, so that we can compare it later with other datasets.

The authors started by choosing an inventory of semantic relations. They aimed for such a set of relations that is exhaustive (enables the description of relations between any pair of nominals) and mutually exclusive (given the context and the pair of nominals, only one relation should be selectable).  Chosen relations with descriptions and examples are listed in \autoref{table01:S10T8}. 

They decided to accept as relation arguments any noun phrases with common-noun heads, not just for example named entities mentioning: \vuvozovkach{Named entities are a specific category of nominal expressions best dealt with using techniques which do not apply to common nouns.} They restricted noun phrases to single words with the exception to lexicalized terms (such as science fiction).

The annotation process consisted of three rounds. In the first round, authors manually collected around \num{1200} sentences for each relation through pattern-based Web search (with at least a hundred patterns per relation). This way, they obtained around \num{1200} sentences for each relation. In the second round, each sentence was annotated by two independent annotators. In the third round, disagreements were resolved, and the dataset was finalized. Every sentence was classified either as a true relation mention or was a near-miss. The near-miss sentences were classified as \relationtype{other}, or were removed.

The relations inventory constains nine positive relations and one positive. The authors decided to include the directionality into the relation a therefore the inventory size is $9 \cdot 2 + 1$ in total. 

The dataset contains \num{10717} relation mentions. For the original competition, teams were given three training dataset of sizes \num{1000} (TD1), \num{2000} (TD2), \num{4000} (TD3), and \num{8000} (TD4). Since there was a notable gain TD3 →TD4, the authors concluded that even larger dataset might be helpful to increase the performance of models. On the topic, the creators have written:

\begin{quotation}.. that is so much easier said than done: it took the organizers well in excess of \num{1000} person-hours to pin down the problem, hone the guidelines and
relation definitions, construct sufficient amounts of trustworthy training data, and run the task.
\end{quotation}


\begin{table}



\caption{S10T8 summary. List of relations, their official descriptions, a random relation mention and both the relative and the absolute count of mentions.}

\label{table01:S10T8}

\begin{tabular}{p{12,2cm} P{1,3cm} }



\hline
\hline
\relationcell{Cause-Effect}{An event or object leads to an effect.}{The \underline{burst} has been caused by water hammer \underline{\smash{pressure}}.} & \freqencycell{12.4}{1331}   \\ 
\hline 
\relationcell{Instrument-Agency}{An agent uses an instrument. }{The \underline{author} of a keygen uses a \underline{disassembler} to look at the raw assembly code.}  & \freqencycell{6.2}{660}   \\ 
\hline 
\relationcell{Product-Producer}{A producer causes a product to exist.}{The \underline{\smash{factory}}'s products have included flower pots, Finnish rooster-whistles, pans, \underline{\smash{trays}}, tea pots, ash trays and air moisturisers.}  & \freqencycell{8.8}{948}   \\ 
\hline 
\relationcell{Content-Container}{An object is physically stored in a delineated area of space.}{This cut blue and white striped cotton \underline{dress} with red bands on the bodice was in a \underline{trunk} of vintage Barbie clothing.}& \freqencycell{6.8}{732} \\ 
\hline 
\relationcell{Entity-Origin}{An entity is coming or is derived from an origin (e.g., position or material). }{The \underline{avalanches} originated in an extensive \underline{mass} of rock that had previously been hydrothermally altered in large part to clay.} & \freqencycell{9.1}{974}  \\ 
\hline 
\relationcell{Entity-Destination}{An entity is moving towards a destination.}{This book has transported \underline{readers} into \underline{ancient times}.}& \freqencycell{10.6}{1137} \\ 
\hline 
\relationcell{Component-Whole}{ An object is a component of a larger whole.}{The system as described above has its greatest application in an arrayed \underline{configuration} of antenna \underline{elements}}.&  \freqencycell{11.7}{1253} \\ 
\hline 
\relationcell{Member-Collection}{ A member forms a nonfunctional part of a collection}{The \underline{student} \underline{association} is the voice of the undergraduate student population of the State University of New York at Buffalo.} & \freqencycell{8.6}{923}  \\ 
\hline 
\relationcell{Message-Topic}{ A message, written or spoken, is about a topic. }{Cieply's \underline{\smash{story}} makes a compelling \underline{\smash{point}} about modern-day studio economics.} & \freqencycell{8.4}{895}  \\ 
\hline 
\relationcell{Other}{} {The \underline{child} was carefully wrapped and bound into the \underline{cradle} by means of a cord.} & \freqencycell{17.4}{1864}  \\ 
\hline 
\end{tabular} 



\end{table}



\section{TACRED dataset}
The TAC Relation Extraction Dataset was introduced in \citep{zhang2017tacred}. TACRED is a supervised dataset obtained via crowdsourcing. It contains about \num{100000} examples, which makes it about ten times bigger than S10T8 dataset. 

The authors are relatively brief about the data collection process:

\begin{quote}
We create TACRED based on query entities and annotated system responses in the yearly TAC KBP evaluations. ... We make use of Mechanical Turk to annotate each sentence in the source corpus that contains one of these query entities. For each sentence, we ask crowd workers to annotate both the subject and object entity spans and the relation types.
\end{quote}

TACRED relation inventory captures 41 relations with the subject being an organization or a person; plus a negative relation. Objects are of the following types: cause of death, city, country, criminal charge, date, duration, ideology, location, misc (used for alternative name relation and \relationtype{no relation} only), nationality, number, organization, person, religion, state or province, title and URL. The choice of subjects and objects it therefore very different from the S10T8 dataset.


TACRED was designed to be highly unbalanced. 79.5\% of mentions represents the  \relationtype{no relation} relation. This ratio of negative relation should be closer to real-world text and supposedly should help avoid false-positive predictions. However, even if we look only at positive relations, there are vast differences in frequency: the top six relations make up half the dataset and the bottom six less than 2\%. In absolute numbers, the least common \relationtype{ord:dissolved} relation has only 33 mentions, and the median is only 286 mentions.
 

\begin{longtable}{p{12,2cm} P{1,3cm} }

\caption{TACRED summary. List of relations, their official descriptions, a random example and both relative and absolute count. The table is restricted to \relationtype{org:*} relations.}\\

\hline
\hline
\relationcelltacred{no\_relation}  {`` \underline{\smash{One}} step at a time , '' said Con Edison spokesman Chris Olert in Sunday editions of The \underline{\smash{Daily News}} .} & \freqencycell{79.5}{84490}   \\ 
\hline
\relationcelltacred{org:alternate\_names}  {The ARMM was established as a result of the peace agreement between the government and the \underline{\smash{Moro National Liberation Front}} -LRB- \underline{\smash{MNLF}} -RRB- in 1996 .} & \freqencycell{1.3}{1358}   \\ 
\hline
\relationcelltacred{org:city\_of\_headquarters}  {Once completed , the cuts will leave the \underline{\smash{Irvine}} , California-based \underline{\smash{Option One}} subsidiary with about 1,400 employees .} & \freqencycell{0.5}{572}   \\ 
\hline
\relationcelltacred{org:country\_of\_headquarters}  {The Review based its report on a new survey conducted by the \underline{\smash{International Agency for Research on Cancer}} in Lyon , \underline{\smash{France}} .} & \freqencycell{0.7}{752}   \\ 
\hline
\relationcelltacred{org:dissolved}  {News Corp. sold its satellite television service \underline{\smash{DirecTV}} in \underline{\smash{2008}} to Liberty Media .} & \freqencycell{0.0}{32}   \\ 
\hline
\relationcelltacred{org:founded}  {New York-based \underline{\smash{Zirh}} was founded in \underline{\smash{1995}} and makes products using natural oils and extracts .} & \freqencycell{0.2}{165}   \\ 
\hline
\relationcelltacred{org:founded\_by}  {The \underline{\smash{Jerusalem Foundation}} , a charity founded by \underline{\smash{Kollek}} 40 years ago , said he died of natural causes Tuesday morning .} & \freqencycell{0.3}{267}   \\ 
\hline
\relationcelltacred{org:member\_of}  {Lyons and the \underline{\smash{Red Sox}} say they are n't aware of any other \underline{\smash{Major League Baseball}} team with such an arrangement .} & \freqencycell{0.2}{170}   \\ 
\hline
\relationcelltacred{org:members}  {The NFL refused to abandon the city , and the \underline{\smash{Saints}} won the \underline{\smash{NFC South}} in 2006 , their first season with Brees and Payton .} & \freqencycell{0.3}{285}   \\ 
\hline
\relationcelltacred{org:number\_of\_employees/members}  {Established in September 1969 , the \underline{\smash{organization}} now has \underline{\smash{57}} member states worldwide .} & \freqencycell{0.1}{120}   \\ 
\hline
\relationcelltacred{org:parents}  {The initial offering of AIA raised \$ 178 billion for AIG , while the sale of \underline{\smash{ALICO}} to \underline{\smash{MetLife}} reaped about \$ 155 billion .} & \freqencycell{0.4}{443}   \\ 
\hline
\relationcelltacred{org:political/religious\_affiliation}  {Manila signed a peace treaty with the \underline{\smash{MNLF}} in 1996 , ending a decades-old separatist campaign in return for limited \underline{\smash{Muslim}} self-rule .} & \freqencycell{0.1}{124}   \\ 
\hline
\relationcelltacred{org:shareholders}  {Stop the NAACP and \underline{\smash{Al Sharpton}} 's \underline{\smash{National Action Network}} from committing this disgrace in our community .} & \freqencycell{0.1}{143}   \\ 
\hline
\relationcelltacred{org:stateorprovince\_of\_headquarters}  {Learn More \underline{\smash{Chelsea District Library}} 221 S Main St Chelsea , \underline{\smash{MI}} 48118 -LRB- 734 -RRB- - 475-8732 Find it on a map} & \freqencycell{0.3}{349}   \\ 
\hline
\relationcelltacred{org:subsidiaries}  {The new law will also enable the government to take over \underline{\smash{Austral Lineas Aereas}} , an \underline{\smash{Aerolineas Argentinas}} subsidiary .} & \freqencycell{0.4}{452}   \\ 
\hline
\relationcelltacred{org:top\_members/employees}  {Earlier this year , \underline{\smash{Anatoly Isaikin}} , head of \underline{\smash{Rosoboronexport}} , said Russia still considers Iran a valuable arms customer .} & \freqencycell{2.6}{2769}   \\ 
\hline
\relationcelltacred{org:website}  {\underline{\smash{Swiss Bankers Association}} : \underline{\smash{http://www.swissbanking.org}}} & \freqencycell{0.2}{222}   \\ 
\hline

\iffalse
\relationcelltacred{per:age}  {Doctor \underline{\smash{Carolyn Goodman}} , Rights Champion , Dies at \underline{\smash{91}}} & \freqencycell{0.8}{832}   \\ 
\hline
\relationcelltacred{per:alternate\_names}  {\underline{\smash{Remy Ma}} , whose real name is \underline{\smash{Remy Smith}} , is charged with first - degree assault and other charges .} & \freqencycell{0.1}{152}   \\ 
\hline
\relationcelltacred{per:cause\_of\_death}  {The cause was \underline{\smash{kidney failure}} , said a spokesman for the \underline{\smash{Ali Akbar}} College of Music .} & \freqencycell{0.3}{336}   \\ 
\hline
\relationcelltacred{per:charges}  {Actor \underline{\smash{Danny Glover}} has been convicted in Canada for \underline{\smash{trespassing}} in a hotel during a union rally in 2006 .} & \freqencycell{0.3}{279}   \\ 
\hline
\relationcelltacred{per:children}  {\underline{\smash{Al-Hakim}} 's son , \underline{\smash{Ammar al-Hakim}} , has been groomed for months to take his father 's place .} & \freqencycell{0.3}{346}   \\ 
\hline
\relationcelltacred{per:cities\_of\_residence}  {As part of a Navy family , \underline{\smash{she}} also lived in Long Beach , Calif. , San Diego and \underline{\smash{Annapolis}} .} & \freqencycell{0.7}{741}   \\ 
\hline
\relationcelltacred{per:city\_of\_birth}  {\underline{\smash{Jane Matilda Bolin}} was born on April 11 , 1908 , in \underline{\smash{Poughkeepsie}} , NY .} & \freqencycell{0.1}{102}   \\ 
\hline
\relationcelltacred{per:city\_of\_death}  {The statement was confirmed by publicist Maureen O'Connor , who said \underline{\smash{Dio}} died in \underline{\smash{Los Angeles}} .} & \freqencycell{0.2}{226}   \\ 
\hline
\relationcelltacred{per:countries\_of\_residence}  {His wife , who accompanied Yoadimnadji to Paris , will repatriate \underline{\smash{his}} body to \underline{\smash{Chad}} , the ambassador said .} & \freqencycell{0.8}{818}   \\ 
\hline
\relationcelltacred{per:country\_of\_birth}  {CARACAS , Jan 10 -LRB- Xinhua -RRB- \underline{\smash{Hugo Chavez}} , was born on July 28 , 1954 , in \underline{\smash{Venezuela}} 's Sabaneta .} & \freqencycell{0.0}{52}   \\ 
\hline
\relationcelltacred{per:country\_of\_death}  {Egypt 's state-owned Middle East News Agency said \underline{\smash{Tantawi}} died in \underline{\smash{Saudi Arabia}} , where he attended a religious ceremony .} & \freqencycell{0.1}{60}   \\ 
\hline
\relationcelltacred{per:date\_of\_birth}  {Antonioni was born in \underline{\smash{1912}} in the northern Italian city of \underline{\smash{Ferrara}} .} & \freqencycell{0.1}{102}   \\ 
\hline
\relationcelltacred{per:date\_of\_death}  {\underline{\smash{December 6 , 2007 \underline{\smash{}} Jefferson DeBlanc}} , Hero Pilot , Dies at 86 By RICHARD GOLDSTEIN} & \freqencycell{0.4}{393}   \\ 
\hline
\relationcelltacred{per:employee\_of}  {\underline{\smash{He}} and his group also joined in a legal battle challenging the \underline{\smash{Washington Redskins}} ' trademarked name .} & \freqencycell{2.0}{2162}   \\ 
\hline
\relationcelltacred{per:origin}  {French media are reporting that \underline{\smash{French}} tennis player \underline{\smash{Mathieu Montcourt}} had died at the age of 24 .} & \freqencycell{0.6}{666}   \\ 
\hline
\relationcelltacred{per:other\_family}  {In the interview \underline{\smash{Cunningham}} acknowledged the fragility of \underline{\smash{his}} choreographic record .} & \freqencycell{0.3}{318}   \\ 
\hline
\relationcelltacred{per:parents}  {The outgoing governor of Barinas is \underline{\smash{Hugo de los Reyes Chavez}} , father of \underline{\smash{Hugo}} and Adan Chavez .} & \freqencycell{0.3}{295}   \\ 
\hline
\relationcelltacred{per:religion}  {\underline{\smash{He}} closed out the quarter making seven payments to \underline{\smash{Scientology}} groups totaling \$ 13,500 .} & \freqencycell{0.1}{152}   \\ 
\hline
\relationcelltacred{per:schools\_attended}  {\underline{\smash{She}} graduated from \underline{\smash{Mount Holyoke College}} in 1941 and from the Yale School of Law in 1948 .} & \freqencycell{0.2}{228}   \\ 
\hline
\relationcelltacred{per:siblings}  {\underline{\smash{Raul Castro}} , \underline{\smash{Fidel}} 's younger brother , has made several overtures toward Washington .} & \freqencycell{0.2}{249}   \\ 
\hline
\relationcelltacred{per:spouse}  {After returning to Dothan in 1946 , \underline{\smash{Flowers}} married \underline{\smash{Mary Catherine Russell}} .} & \freqencycell{0.5}{482}   \\ 
\hline
\relationcelltacred{per:stateorprovince\_of\_birth}  {\underline{\smash{Thomas Joseph Meskill}} Jr was born in New Britain , \underline{\smash{Conn}} , on Jan 30 , 1928 .} & \freqencycell{0.1}{71}   \\ 
\hline
\relationcelltacred{per:stateorprovince\_of\_death}  {Jessica Weiner says \underline{\smash{Greenwich}} died of a heart attack at St. Luke 's Roosevelt Hospital in \underline{\smash{New York}} .} & \freqencycell{0.1}{103}   \\ 
\hline
\relationcelltacred{per:stateorprovinces\_of\_residence}  {Sen. \underline{\smash{Chris Dodd}} of \underline{\smash{Connecticut}} has proposed taxing polluters for their carbon emissions .} & \freqencycell{0.5}{483}   \\ 
\hline
\relationcelltacred{per:title}  {\underline{\smash{He}} is the \underline{\smash{founder}} and leader of Architects and Engineers for 9/11 Truth -LRB- AE911Truthorg -RRB- .} & \freqencycell{3.6}{3861}   \\ 
\hline
\fi

\label{table02:tacred}
\end{longtable} 



\section{Riedel NYT dataset}
The previous two datasets were obtained through a tedious human labour -- human annotators went through texts and manually annotated the data. This process is slow and expensive, which explains the relatively small data volume of the datasets. In this section, we introduce a dataset presented in \citep{nytdistant} that was created without the need for any additional manual annotation.


This dataset was generated with the distant supervision approach. This approach is based on aligning structured data (knowledge base) with text, i.e. automatically tagging mentions of the structured data in text. In distant supervision, we usually expect that if there are two entity mentions in a sentence that are related, then the sentence expresses their relationship. The authors acknowledge that this assumption is often violated and they propose a methodology that attempts to predict whether the assumption is violated in a sentence. Using this methodology, they generated a dataset from the The New York Times Annotated Corpus \citep{linguistic2008new} and Freebase \citep{Bollacker08freebase}. We will refer to this dataset as Riedel NYT.


Relation inventory for the usual training part of the dataset contains 58 relations. The best represented is the \relationtype{NA} relation with over 80\%. The representation of relations varies between the train and test set. For example, two relations are present only in the test set. 


\begin{table}



\caption{NYT Riedel summarization.}

\label{table03:nyt}

\begin{tabular}{p{12,2cm} P{1,3cm} }
\hline
\hline
\relationcelltacred{/location/location/contains} {But John Traugott , 68 , a hospital chaplain in \underline{\smash{ Rockaway Park }} , \underline{\smash{ Queens }} hinted at some of Chinatown 's problems .} & \freqencycell{8.6}{58625}   \\ 
\hline
\relationcelltacred{/people/person/nationality} {We were unable to reach agreement , '' Foreign Minister \underline{\smash{ Frank-Walter Steinmeier }} of \underline{\smash{ Germany }} announced tersely to reporters .} & \freqencycell{1.5}{10464}   \\ 
\hline
\relationcelltacred{/people/person/place\_lived} {Camane -LRB- Emily Blunt -RRB- runs around \underline{\smash{ Rome }} putting up signs declaring that Octavius is \underline{\smash{ Julius Caesar }} 's rightful heir .} & \freqencycell{1.2}{8275}   \\ 
\hline
\relationcelltacred{/business/company/founders} {\underline{\smash{ Nick Grouf }} , president and chief executive at \underline{\smash{ Spot Runner }} in Los Angeles , is scheduled to announce the investments today .} & \freqencycell{0.1}{999}   \\ 
\hline
\relationcelltacred{/people/deceased\_person/place\_of\_death} {\underline{\smash{ Marie Antoinette }} finally did arrive in \underline{\smash{ Paris }} , at Christian Dior , where the designer John Galliano proclaimed her his platinum muse .} & \freqencycell{0.3}{2190}   \\ 
\hline
\relationcelltacred{/business/person/company} {\underline{\smash{ Christopher Bailey }} , with his light-handed take on \underline{\smash{ Burberry }} 's heritage , could add some military backbone and useful outerwear .} & \freqencycell{0.9}{6455}   \\ 
\hline
\relationcelltacred{/location/us\_county/county\_seat} {Mr. Perhacs was taken to \underline{\smash{ Jersey City }} Medical Center , where he died , said Edward J. De Fazio , the \underline{\smash{ Hudson County }} prosecutor .} & \freqencycell{0.0}{125}   \\ 
\hline
\relationcelltacred{/business/company/place\_founded} {Next month in \underline{\smash{ Paris }} , Ms. Tilbury will direct makeup at the spring fashion shows of Lanvin , \underline{\smash{ Chloé }} and Alexander McQueen .} & \freqencycell{0.1}{537}   \\ 
\hline
\relationcelltacred{/people/person/place\_of\_birth} {\underline{\smash{ Preston Robert Tisch }} was born in the Bensonhurst section of \underline{\smash{ Brooklyn }} on April 29 , 1926 , to parents who came from Russia .} & \freqencycell{0.5}{3603}   \\ 
\hline
\relationcelltacred{/film/film/featured\_film\_locations} {\underline{\smash{ Half Nelson }} , '' a new independent film about an idealistic young \underline{\smash{ Brooklyn }} teacher , takes this claim at face value . ''} & \freqencycell{0.0}{19}   \\ 
\hline
\relationcelltacred{/people/person/children} {Were \underline{\smash{ David }} and \underline{\smash{ Solomon }} really kings of a state with growing power in the 10th century B.C. ?} & \freqencycell{0.1}{543}   \\ 
\hline
\relationcelltacred{/location/neighborhood/neighborhood\_of} {NEW YORK LIKE A NATIVE Sunday at 1:30 p.m. , '' Fort Greene and \underline{\smash{ Clinton Hill }} , '' a \underline{\smash{ Brooklyn }} tour ; \$ 15 .} & \freqencycell{0.9}{6056}   \\ 
\hline
\relationcelltacred{/location/country/administrative\_divisions} {Real Estate in NYC , he managed and developed properties in Manhattan , throughout the country ; and \underline{\smash{ Paris }} , \underline{\smash{ France }} .} & \freqencycell{1.1}{7451}   \\ 
\hline
\relationcelltacred{/location/country/capital} {Like countless ambitious young men from southern India , Mr. Narayanan then traveled north to \underline{\smash{ New Delhi }} , \underline{\smash{ India }} 's capital .} & \freqencycell{1.3}{8614}   \\ 
\hline
\relationcelltacred{/people/ethnicity/included\_in\_group} {With the city so influenced by Chinese and \underline{\smash{ Japanese }} culture , \underline{\smash{ Asian }} cuisine is always an excellent bet .} & \freqencycell{0.0}{6}   \\ 
\hline
\relationcelltacred{/people/place\_of\_interment/interred\_here} {\underline{\smash{ Franklin D. Roosevelt }} , paralyzed from polio , lulled himself to sleep by imagining himself as a boy sledding down a snowy slope at \underline{\smash{ Hyde Park }} .} & \freqencycell{0.0}{31}   \\ 
\hline
\relationcelltacred{/location/administrative\_division/country} {Meier , a \underline{\smash{ Moscow }} correspondent for Time magazine from 1996 to 2001 , recounts his travels across post-Soviet \underline{\smash{ Russia }} .} & \freqencycell{1.1}{7448}   \\ 
\hline
\relationcelltacred{/time/event/locations} {\underline{\smash{ New Orleans }} is a town where generally it helps to have local roots that go back at least one or two generations , if not back to the days before the \underline{\smash{ Louisiana Purchase }} .} & \freqencycell{0.0}{8}   \\ 
\hline
\relationcelltacred{/location/de\_state/capital} {We 're fighting for our historic role , '' said Karl Peter Bruch , a state secretary of \underline{\smash{ Rhineland-Palatinate }} , of which \underline{\smash{ Mainz }} is the capital . ''} & \freqencycell{0.0}{7}   \\ 
\hline
\relationcelltacred{/location/us\_state/capital} {Yeah , you can have it back , '' Acting Gov. Richard J. Codey of \underline{\smash{ New Jersey }} said by phone from \underline{\smash{ Trenton }} .} & \freqencycell{0.1}{764}   \\ 
\hline
\relationcelltacred{/business/company\_advisor/companies\_advised} {In 2006 , the annual meeting morphed into a three and a half hour celebration of \underline{\smash{ Sanford I. Weill }} , \underline{\smash{ Citigroup }} 's departing chairman .} & \freqencycell{0.0}{10}   \\ 
\hline
\relationcelltacred{/people/person/religion} {Wesley , the inner circle closest to the tabernacle , was named for \underline{\smash{ John Wesley }} , the English clergyman who founded \underline{\smash{ Methodism }} .} & \freqencycell{0.0}{176}   \\ 
\hline
\relationcelltacred{/people/deceased\_person/place\_of\_burial} {\underline{\smash{ Franklin D. Roosevelt }} , paralyzed from polio , lulled himself to sleep by imagining himself as a boy sledding down a snowy slope at \underline{\smash{ Hyde Park }} .} & \freqencycell{0.0}{31}   \\ 
\hline
\relationcelltacred{/people/person/ethnicity} {The Senate has approved a measure by Senator \underline{\smash{ John McCain }} , Republican of Arizona , i to ban abusive treatment of prisoners in \underline{\smash{ American }} custody .} & \freqencycell{0.0}{175}   \\ 
\hline
\relationcelltacred{/sports/sports\_team/location} {The Pacers are headed back to \underline{\smash{ Auburn Hills }} , Mich. , for a conference semifinal series against the defending champion \underline{\smash{ Detroit Pistons }} .} & \freqencycell{0.0}{233}   \\ 
\hline
\relationcelltacred{/broadcast/content/location} {Mr. Coleman , the former deputy director , is now general manager of \underline{\smash{ WDET }} , a \underline{\smash{ Detroit }} public radio station .} & \freqencycell{0.0}{8}   \\ 
\hline
\relationcelltacred{/film/film\_festival/location} {On Oct. 9 , it will be shown at the \underline{\smash{ Raindance Film Festival }} in \underline{\smash{ London }} .} & \freqencycell{0.0}{4}   \\ 
\hline
\relationcelltacred{/location/it\_region/capital} {Born in Milan in 1922 , she was raised in \underline{\smash{ Tuscany }} near \underline{\smash{ Florence }} , where she first developed her love of nature .} & \freqencycell{0.0}{22}   \\ 
\hline
\relationcelltacred{/business/shopping\_center\_owner/shopping\_centers\_owned} {Earlier this week , the company said it expected to sell \underline{\smash{ Madrid Xanadú }} and its half-interest in two other malls , Vaughan Mills in Ontario and St. Enoch Centre in Glasgow , to \underline{\smash{ Ivanhoe Cambridge }} , a Montreal company that is Mills 's partner in the Canadian and Scottish properties .} & \freqencycell{0.0}{1}   \\ 
\hline
\relationcelltacred{/people/person/profession} {It definitely has to be something different , '' said \underline{\smash{ Tom Fulp }} , president and \underline{\smash{ Webmaster }} of newgrounds.com . ''} & \freqencycell{0.0}{8}   \\ 
\hline
\relationcelltacred{/business/company/major\_shareholders} {\underline{\smash{ Warren Buffett }} , a true god of investments , says much the same thing eloquently in his most recent annual report for \underline{\smash{ Berkshire Hathaway }} .} & \freqencycell{0.1}{352}   \\ 
\hline
\relationcelltacred{/location/in\_state/legislative\_capital} {A Mobile Society For 15 years , Vilas Jaganath Kamkar had been taking the bus from his village in \underline{\smash{ Maharashtra }} state to \underline{\smash{ Mumbai }} , its capital , where he worked as a taxicab driver .} & \freqencycell{0.0}{4}   \\ 
\hline
\relationcelltacred{/location/in\_state/administrative\_capital} {A Mobile Society For 15 years , Vilas Jaganath Kamkar had been taking the bus from his village in \underline{\smash{ Maharashtra }} state to \underline{\smash{ Mumbai }} , its capital , where he worked as a taxicab driver .} & \freqencycell{0.0}{4}   \\ 
\hline
\relationcelltacred{/business/business\_location/parent\_company} {Even \underline{\smash{ Sweden }} got tough last week , dispatching a former chief of the insurer \underline{\smash{ Skandia }} to prison for two years .} & \freqencycell{0.0}{18}   \\ 
\hline
\relationcelltacred{/people/family/members} {It is particularly appropriate for the \underline{\smash{ Guggenheim }} to be organizing the show because it was \underline{\smash{ Peggy Guggenheim }} who gave Pollock his first one-man exhibition , in 1943 .} & \freqencycell{0.0}{4}   \\ 
\hline
\relationcelltacred{/location/jp\_prefecture/capital} {In \underline{\smash{ Sapporo }} , \underline{\smash{ Hokkaido }} 's capital , Mr. Abe collected signatures in a campaign to save draft-horse racing . ''} & \freqencycell{0.0}{2}   \\ 
\hline
\relationcelltacred{/film/film\_location/featured\_in\_films} {\underline{\smash{ Half Nelson }} -- Ryan Gosling as a \underline{\smash{ Brooklyn }} schoolteacher , trying to balance his commitment to his underprivileged students with his serious drug and alcohol problems .} & \freqencycell{0.0}{20}   \\ 
\hline
\relationcelltacred{/people/family/country} {Even after \underline{\smash{ Iraq }} gained its independence from Britain in 1932 , the club -- like Iraq 's \underline{\smash{ Hashemite }} monarchy -- was viewed as a foreign presence by many Iraqis .} & \freqencycell{0.0}{6}   \\ 
\hline
\relationcelltacred{/business/company/locations} {Even \underline{\smash{ Sweden }} got tough last week , dispatching a former chief of the insurer \underline{\smash{ Skandia }} to prison for two years .} & \freqencycell{0.0}{18}   \\ 
\hline
\relationcelltacred{/people/ethnicity/includes\_groups} {With the city so influenced by Chinese and \underline{\smash{ Japanese }} culture , \underline{\smash{ Asian }} cuisine is always an excellent bet .} & \freqencycell{0.0}{6}   \\ 
\hline
\relationcelltacred{/business/company/advisors} {\underline{\smash{ Sanford I. Weill }} , who stepped down as \underline{\smash{ Citigroup }} 's chairman last week , has always been a voracious reader .} & \freqencycell{0.0}{56}   \\ 
\hline
\relationcelltacred{/people/profession/people\_with\_this\_profession} {A sports article on the Spotlight page on Sunday about \underline{\smash{ Dick Bavetta }} , a longtime referee in the \underline{\smash{ National Basketball Association }} , misstated the number he was approaching to set the record for regular-season games worked .} & \freqencycell{0.0}{2}   \\ 
\hline
\relationcelltacred{/location/br\_state/capital} {José Bezerra da Silva was born into poverty on March 9 , 1927 , in \underline{\smash{ Recife }} , the capital of the northeastern state of \underline{\smash{ Pernambuco }} .} & \freqencycell{0.0}{6}   \\ 
\hline
\relationcelltacred{/location/cn\_province/capital} {That is one reason that \underline{\smash{ Hunan }} 's fast-growing provincial capital , \underline{\smash{ Changsha }} , is beginning to siphon some workers back from Guangdong .} & \freqencycell{0.0}{2}   \\ 
\hline
\relationcelltacred{/broadcast/producer/location} {The Saudi ambassador in \underline{\smash{ London }} , Prince Turki al-Faisal , told \underline{\smash{ BBC }} television on Sunday that it was '' premature '' to speak of such a link . ''} & \freqencycell{0.0}{69}   \\ 
\hline
\relationcelltacred{/location/fr\_region/capital} {A perfect expression of the island 's historic contradictions is the headless marble statue , in \underline{\smash{ Fort-de-France }} , of Marie-Jos èphe - Rose Tascher de la Pagerie , the most famous daughter of \underline{\smash{ Martinique }} , who became Napoleon 's Empress Josephine .} & \freqencycell{0.0}{1}   \\ 
\hline
\relationcelltacred{/people/ethnicity/geographic\_distribution} {In Ottawa on Monday , the \underline{\smash{ Russian }} foreign minister , Sergey V. Lavrov , again floated a proposal to enrich uranium inside \underline{\smash{ Russia }} for use by Iran .} & \freqencycell{0.1}{744}   \\ 
\hline
\relationcelltacred{/location/province/capital} {The Bean Red Ribbon Pairs was won by Chris Buchanon and B. J. Trelford of \underline{\smash{ Edmonton }} , \underline{\smash{ Alberta }} .} & \freqencycell{0.0}{47}   \\ 
\hline
\relationcelltacred{/location/in\_state/judicial\_capital} {A Mobile Society For 15 years , Vilas Jaganath Kamkar had been taking the bus from his village in \underline{\smash{ Maharashtra }} state to \underline{\smash{ Mumbai }} , its capital , where he worked as a taxicab driver .} & \freqencycell{0.0}{3}   \\ 
\hline
\relationcelltacred{/business/shopping\_center/owner} {Earlier this week , the company said it expected to sell \underline{\smash{ Madrid Xanadú }} and its half-interest in two other malls , Vaughan Mills in Ontario and St. Enoch Centre in Glasgow , to \underline{\smash{ Ivanhoe Cambridge }} , a Montreal company that is Mills 's partner in the Canadian and Scottish properties .} & \freqencycell{0.0}{1}   \\ 
\hline
\relationcelltacred{/location/mx\_state/capital} {In the meantime , there are growing signs that the serial-style killings have spread to other cities , like Chihuahua , 200 miles along the border ; Toluca , a suburb of Mexico City ; the Gulf Coast capital of Veracruz ; and \underline{\smash{ Tuxtla Gutiérrez }} in the southern state of \underline{\smash{ Chiapas }} .} & \freqencycell{0.0}{1}   \\ 
\hline
\relationcelltacred{NA} {Op-Ed Contributor \underline{\smash{ John Catsimatidis }} is the chairman and chief executive of a chain of \underline{\smash{ New York City }} supermarkets .} & \freqencycell{81.7}{557819}   \\ 
\hline
\relationcelltacred{/sports/sports\_team\_location/teams} {The Pacers are headed back to \underline{\smash{ Auburn Hills }} , Mich. , for a conference semifinal series against the defending champion \underline{\smash{ Detroit Pistons }} .} & \freqencycell{0.0}{223}   \\ 
\hline
\relationcelltacred{/people/ethnicity/people} {The protests that unfolded in provincial \underline{\smash{ Kyrgyz }} cities spread swiftly , ultimately forcing that country 's president , \underline{\smash{ Askar Akayev }} , to flee .} & \freqencycell{0.0}{162}   \\ 
\hline
\relationcelltacred{/business/company\_shareholder/major\_shareholder\_of} {\underline{\smash{ Warren Buffett }} , a true god of investments , says much the same thing eloquently in his most recent annual report for \underline{\smash{ Berkshire Hathaway }} .} & \freqencycell{0.0}{306}   \\ 
\hline
\relationcelltacred{/business/company/industry} {Already this year , America 's biggest department store chain , \underline{\smash{ Federated Department Stores }} , has bought the second biggest , May \underline{\smash{ Department Stores }} , in a deal valued at \$ 16 billion .} & \freqencycell{0.0}{6}   \\ 
\hline
\relationcelltacred{/location/country/languages\_spoken} {Even during summer vacations in \underline{\smash{ India }} , there were barriers that prevented us from speaking freely : my \underline{\smash{ Telugu }} was less than perfect ; my grandmother 's English was never smooth .} & \freqencycell{0.0}{3}   \\ 
\hline
\relationcelltacred{/base/locations/countries/states\_provinces\_within} {Over the course of my reign and certainly since I first visited Jamestown in 1957 , my country has become a much more diverse society , just as the commonwealth of \underline{\smash{ Virginia }} and the whole \underline{\smash{ United States of America }} have also undergone a major social change , '' the queen said . ''} & \freqencycell{0.0}{1}   \\ 
\hline


\end{tabular} 



\end{table}


